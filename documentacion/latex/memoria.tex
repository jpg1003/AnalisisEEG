\documentclass[a4paper,12pt,twoside]{memoir}

% Castellano
\usepackage[spanish,es-tabla]{babel}
\selectlanguage{spanish}
\usepackage[utf8]{inputenc}
\usepackage[T1]{fontenc}
\usepackage{lmodern} % Scalable font
\usepackage{microtype}
\usepackage{placeins}

\RequirePackage{booktabs}
\RequirePackage[table]{xcolor}
\RequirePackage{xtab}
\RequirePackage{multirow}

% Links
\PassOptionsToPackage{hyphens}{url}\usepackage[colorlinks]{hyperref}
\hypersetup{
	allcolors = {red}
}

% Ecuaciones
\usepackage{amsmath}

% Rutas de fichero / paquete
\newcommand{\ruta}[1]{{\sffamily #1}}

% Párrafos
\nonzeroparskip

% Huérfanas y viudas
\widowpenalty100000
\clubpenalty100000

% Imágenes

% Comando para insertar una imagen en un lugar concreto.
% Los parámetros son:
% 1 --> Ruta absoluta/relativa de la figura
% 2 --> Texto a pie de figura
% 3 --> Tamaño en tanto por uno relativo al ancho de página
\usepackage{graphicx}
\newcommand{\imagen}[3]{
	\begin{figure}[!h]
		\centering
		\includegraphics[width=#3\textwidth]{#1}
		\caption{#2}\label{fig:#1}
	\end{figure}
	\FloatBarrier
}

% Comando para insertar una imagen sin posición.
% Los parámetros son:
% 1 --> Ruta absoluta/relativa de la figura
% 2 --> Texto a pie de figura
% 3 --> Tamaño en tanto por uno relativo al ancho de página
\newcommand{\imagenflotante}[3]{
	\begin{figure}
		\centering
		\includegraphics[width=#3\textwidth]{#1}
		\caption{#2}\label{fig:#1}
	\end{figure}
}

% El comando \figura nos permite insertar figuras comodamente, y utilizando
% siempre el mismo formato. Los parametros son:
% 1 --> Porcentaje del ancho de página que ocupará la figura (de 0 a 1)
% 2 --> Fichero de la imagen
% 3 --> Texto a pie de imagen
% 4 --> Etiqueta (label) para referencias
% 5 --> Opciones que queramos pasarle al \includegraphics
% 6 --> Opciones de posicionamiento a pasarle a \begin{figure}
\newcommand{\figuraConPosicion}[6]{%
  \setlength{\anchoFloat}{#1\textwidth}%
  \addtolength{\anchoFloat}{-4\fboxsep}%
  \setlength{\anchoFigura}{\anchoFloat}%
  \begin{figure}[#6]
    \begin{center}%
      \Ovalbox{%
        \begin{minipage}{\anchoFloat}%
          \begin{center}%
            \includegraphics[width=\anchoFigura,#5]{#2}%
            \caption{#3}%
            \label{#4}%
          \end{center}%
        \end{minipage}
      }%
    \end{center}%
  \end{figure}%
}

%
% Comando para incluir imágenes en formato apaisado (sin marco).
\newcommand{\figuraApaisadaSinMarco}[5]{%
  \begin{figure}%
    \begin{center}%
    \includegraphics[angle=90,height=#1\textheight,#5]{#2}%
    \caption{#3}%
    \label{#4}%
    \end{center}%
  \end{figure}%
}
% Para las tablas
\newcommand{\otoprule}{\midrule [\heavyrulewidth]}
%
% Nuevo comando para tablas pequeñas (menos de una página).
\newcommand{\tablaSmall}[5]{%
 \begin{table}
  \begin{center}
   \rowcolors {2}{gray!35}{}
   \begin{tabular}{#2}
    \toprule
    #4
    \otoprule
    #5
    \bottomrule
   \end{tabular}
   \caption{#1}
   \label{tabla:#3}
  \end{center}
 \end{table}
}

%
% Nuevo comando para tablas pequeñas (menos de una página).
\newcommand{\tablaSmallSinColores}[5]{%
 \begin{table}[H]
  \begin{center}
   \begin{tabular}{#2}
    \toprule
    #4
    \otoprule
    #5
    \bottomrule
   \end{tabular}
   \caption{#1}
   \label{tabla:#3}
  \end{center}
 \end{table}
}

\newcommand{\tablaApaisadaSmall}[5]{%
\begin{landscape}
  \begin{table}
   \begin{center}
    \rowcolors {2}{gray!35}{}
    \begin{tabular}{#2}
     \toprule
     #4
     \otoprule
     #5
     \bottomrule
    \end{tabular}
    \caption{#1}
    \label{tabla:#3}
   \end{center}
  \end{table}
\end{landscape}
}

%
% Nuevo comando para tablas grandes con cabecera y filas alternas coloreadas en gris.
\newcommand{\tabla}[6]{%
  \begin{center}
    \tablefirsthead{
      \toprule
      #5
      \otoprule
    }
    \tablehead{
      \multicolumn{#3}{l}{\small\sl continúa desde la página anterior}\\
      \toprule
      #5
      \otoprule
    }
    \tabletail{
      \hline
      \multicolumn{#3}{r}{\small\sl continúa en la página siguiente}\\
    }
    \tablelasttail{
      \hline
    }
    \bottomcaption{#1}
    \rowcolors {2}{gray!35}{}
    \begin{xtabular}{#2}
      #6
      \bottomrule
    \end{xtabular}
    \label{tabla:#4}
  \end{center}
}

%
% Nuevo comando para tablas grandes con cabecera.
\newcommand{\tablaSinColores}[6]{%
  \begin{center}
    \tablefirsthead{
      \toprule
      #5
      \otoprule
    }
    \tablehead{
      \multicolumn{#3}{l}{\small\sl continúa desde la página anterior}\\
      \toprule
      #5
      \otoprule
    }
    \tabletail{
      \hline
      \multicolumn{#3}{r}{\small\sl continúa en la página siguiente}\\
    }
    \tablelasttail{
      \hline
    }
    \bottomcaption{#1}
    \begin{xtabular}{#2}
      #6
      \bottomrule
    \end{xtabular}
    \label{tabla:#4}
  \end{center}
}

%
% Nuevo comando para tablas grandes sin cabecera.
\newcommand{\tablaSinCabecera}[5]{%
  \begin{center}
    \tablefirsthead{
      \toprule
    }
    \tablehead{
      \multicolumn{#3}{l}{\small\sl continúa desde la página anterior}\\
      \hline
    }
    \tabletail{
      \hline
      \multicolumn{#3}{r}{\small\sl continúa en la página siguiente}\\
    }
    \tablelasttail{
      \hline
    }
    \bottomcaption{#1}
  \begin{xtabular}{#2}
    #5
   \bottomrule
  \end{xtabular}
  \label{tabla:#4}
  \end{center}
}



\definecolor{cgoLight}{HTML}{EEEEEE}
\definecolor{cgoExtralight}{HTML}{FFFFFF}

%
% Nuevo comando para tablas grandes sin cabecera.
\newcommand{\tablaSinCabeceraConBandas}[5]{%
  \begin{center}
    \tablefirsthead{
      \toprule
    }
    \tablehead{
      \multicolumn{#3}{l}{\small\sl continúa desde la página anterior}\\
      \hline
    }
    \tabletail{
      \hline
      \multicolumn{#3}{r}{\small\sl continúa en la página siguiente}\\
    }
    \tablelasttail{
      \hline
    }
    \bottomcaption{#1}
    \rowcolors[]{1}{cgoExtralight}{cgoLight}

  \begin{xtabular}{#2}
    #5
   \bottomrule
  \end{xtabular}
  \label{tabla:#4}
  \end{center}
}



\graphicspath{ {../imagenes/} }



% Capítulos
\chapterstyle{bianchi}
\newcommand{\capitulo}[2]{
	\setcounter{chapter}{#1}
	\setcounter{section}{0}
	\setcounter{figure}{0}
	\setcounter{table}{0}
	\chapter*{\thechapter.\enskip #2}
	\addcontentsline{toc}{chapter}{\thechapter.\enskip #2}
	\markboth{#2}{#2}
}

% Apéndices
\renewcommand{\appendixname}{Apéndice}
\renewcommand*\cftappendixname{\appendixname}

\newcommand{\apendice}[1]{
	%\renewcommand{\thechapter}{A}
	\chapter{#1}
}

\renewcommand*\cftappendixname{\appendixname\ }

% Formato de portada
\makeatletter
\usepackage{xcolor}
\newcommand{\tutor}[1]{\def\@tutor{#1}}
\newcommand{\course}[1]{\def\@course{#1}}
\definecolor{cpardoBox}{HTML}{E6E6FF}
\def\maketitle{
  \null
  \thispagestyle{empty}
  % Cabecera ----------------
\noindent\includegraphics[width=\textwidth]{cabecera}\vspace{1cm}%
  \vfill
  % Título proyecto y escudo informática ----------------
  \colorbox{cpardoBox}{%
    \begin{minipage}{.8\textwidth}
      \vspace{.5cm}\Large
      \begin{center}
      \textbf{TFG del Grado en Ingeniería Informática}\vspace{.6cm}\\
      \textbf{\LARGE\@title{}}
      \end{center}
      \vspace{.2cm}
    \end{minipage}

  }%
  \hfill\begin{minipage}{.20\textwidth}
    \includegraphics[width=\textwidth]{escudoInfor}
  \end{minipage}
  \vfill
  % Datos de alumno, curso y tutores ------------------
  \begin{center}%
  {%
    \noindent\LARGE
    Presentado por \@author{}\\ 
    en Universidad de Burgos --- \@date{}\\
    Tutor: \@tutor{}\\
  }%
  \end{center}%
  \null
  \cleardoublepage
  }
\makeatother

\newcommand{\nombre}{José Luis Pérez Gómez} %%% cambio de comando

% Datos de portada
\title{Análisis de señales EEG}
\author{\nombre}
\tutor{Bruno Baruque Zanón - Jesús Enrique Sierra García}
\date{\today}


\begin{document}

\maketitle


\newpage\null\thispagestyle{empty}\newpage


%%%%%%%%%%%%%%%%%%%%%%%%%%%%%%%%%%%%%%%%%%%%%%%%%%%%%%%%%%%%%%%%%%%%%%%%%%%%%%%%%%%%%%%%
\thispagestyle{empty}


\noindent\includegraphics[width=\textwidth]{cabecera}\vspace{1cm}

\noindent D. Bruno Baruque Zanón, profesor del departamento de Digitalización, área de Ciencia de la Computación e Inteligencia Artificial.

\noindent Expone:

\noindent Que el alumno D. \nombre, con DNI 46878665L, ha realizado el Trabajo final de Grado en Ingeniería Informática titulado Análisis de señales EEG. 

\noindent Y que dicho trabajo ha sido realizado por el alumno bajo la dirección del que suscribe, en virtud de lo cual se autoriza su presentación y defensa.

\begin{center} %\large
En Burgos, {\large \today}
\end{center}

\vfill\vfill\vfill

% Author and supervisor
\begin{minipage}{0.45\textwidth}
\begin{flushleft} %\large
Vº. Bº. del Tutor:\\[2cm]
D. Bruno Baruque Zanón
\end{flushleft}
\end{minipage}
\hfill
\begin{minipage}{0.45\textwidth}
\begin{flushleft} %\large
Vº. Bº. del co-tutor:\\[2cm]
D. Jesús Enrique Sierra García
\end{flushleft}
\end{minipage}
\hfill

\vfill

% para casos con solo un tutor comentar lo anterior
% y descomentar lo siguiente
%Vº. Bº. del Tutor:\\[2cm]
%D. nombre tutor


\newpage\null\thispagestyle{empty}\newpage




\frontmatter

% Abstract en castellano
\renewcommand*\abstractname{Resumen}
\begin{abstract}


El objetivo de este proyecto esta orientado principalmente a la investigación de modelos de aprendizaje para el desarrollo de un sistema de análisis y clasificación de datos EEG (Electroencefalograma) enfocado en la detección y clasificación de intenciones de movimiento de un cursor en la pantalla que tiene como posibles direcciones ir hacia arriba, abajo, derecha e izquierda. 

Utilizando técnicas avanzadas de aprendizaje automático y procesamiento de datos, este sistema puede servir como base para aplicaciones en interfaces cerebro-computadora (BCI). 

Los datos se procesan a través de un flujo de trabajo organizado en un sistema modular, donde se llevan a cabo tareas específicas como el preprocesamiento de datos y el entrenamiento de modelos. El proyecto estudia técnicas como el escalado de datos, la validación cruzada y el uso de redes neuronales avanzadas para asegurar la precisión y fiabilidad de los resultados.


\end{abstract}

\renewcommand*\abstractname{Descriptores}
\begin{abstract}
Señales EEG, Análisis de datos EEG, preprocesamiento de datos, modelos predictivos, redes neuronales, machine learning, minería de datos, tasa de acierto, matriz de confusión, Python, Jupyter Notebook, visualización de datos \ldots
\end{abstract}

\clearpage

% Abstract en inglés
\renewcommand*\abstractname{Abstract}
\begin{abstract}
A \textbf{brief} The objective of this project is mainly oriented to the investigation of learning models for the development of an EEG (Electroencephalogram) data analysis and classification system focused on the detection and classification of movement intentions of a cursor on the screen that has as possible directions up. , down, right and left. 

Using advanced machine learning and data processing techniques, this system can serve as a basis for applications in brain-computer interfaces (BCI). 

Data is processed through a workflow organized in a modular system, where specific tasks such as data preprocessing and model training are carried out. The project studies techniques such as data scaling, cross-validation and the use of advanced neural networks to ensure the accuracy and reliability of the results.
\end{abstract}

\renewcommand*\abstractname{Keywords}
\begin{abstract}
EEG signals, EEG data analysis, data preprocessing, predictive models, neural networks, machine learning, data mining, hit rate, confusion matrix, Python, Jupyter Notebook, data visualization\ldots
\end{abstract}

\clearpage

% Indices
\tableofcontents

\clearpage

\listoffigures

\clearpage

\listoftables
\clearpage

\mainmatter
\capitulo{1}{Introducción}

La Universidad de Burgos, dentro del área de conocimiento de Ingeniería de Sistemas y Automática, dispone de un interfaz BCI (Brain Computer Interface) para la captación de señales cerebrales. 
Empleando ese interfaz se han realizado diferentes experimentos que han permitido recoger información de la actividad cerebral mientras los usuarios ejecutaban diferentes tareas cotidianas. 

Este Trabajo de Fin de Grado (TFG) tiene como objetivo el análisis de la información obtenida en esos experimentos. Se entrenarán diferentes algoritmos para clasificar la acción realizada por el usuario a partir de las señales generadas por el BCI. Con este propósito, se evaluarán diferentes procesamiento de datos, modelos de machine learning y redes neuronales para la clasificación automática de señales.

Los datos aportados son de tipo EEG (Electroencefalograma) son datos referentes a experimentos basados en acciones de individuas sobre las teclas de un teclado: arriba, abajo, izquierda, derecha.

El análisis de estos datos y su evaluación en diferentes algoritmos esta basada en predecir qué teclas del teclado se han pulsado según las señales EEG captadas con la interfaz BCI.



\capitulo{2}{Objetivos del proyecto}

-----------
Este apartado explica de forma precisa y concisa cuales son los objetivos que se persiguen con la realización del proyecto. Se puede distinguir entre los objetivos marcados por los requisitos del software a construir y los objetivos de carácter técnico que plantea a la hora de llevar a la práctica el proyecto.
---------------


El objetivo principal de este proyecto es diseñar, implementar y evaluar un sistema integral de análisis y clasificación de señales EEG (Electroencefalograma) centrado en la detección precisa y la clasificación efectiva de acciones de movimiento. 

Este sistema se construirá utilizando técnicas avanzadas de machine learning y redes neuronales, con el propósito de desarrollar modelos predictivos robustos capaces de interpretar y categorizar las señales cerebrales asociadas con movimientos específicos, como hacia arriba, abajo, derecha e izquierda.

El proyecto se enfocará en varias etapas clave:

\begin{itemize}
\tightlist
\item
\textbf{Adquisición y Preprocesamiento de Datos:}

 Se implementará un flujo de trabajo para la importación, limpieza y estandarización de los datos EEG, asegurando la calidad y consistencia necesarias para el análisis subsiguiente.

\textbf{Desarrollo de Modelos Predictivos:}

 Se explorarán y desarrollarán diferentes modelos de machine learning y redes neuronales, adaptados específicamente para el análisis de señales EEG. Esto incluirá el entrenamiento y la optimización de modelos para mejorar la precisión y la generalización.

\textbf{Validación y Evaluación:} 

Se llevará a cabo una evaluación exhaustiva de los modelos desarrollados, utilizando métricas como la tase de acierto. Además, se emplearán técnicas para garantizar la robustez de los resultados.

\end{itemize}

En resumen, el proyecto tiene como objetivo fundamental avanzar en la comprensión y aplicación de técnicas de machine learning y redes neuronales para el análisis de datos de señales EEG , específicamente en el contexto de para la detección precisa de acciones de movimiento.

\capitulo{3}{Conceptos teóricos}

El proyecto se enfoca en el análisis y clasificación de señales EEG (Electroencefalograma) para detectar intenciones de movimiento mediante técnicas avanzadas de machine learning y redes neuronales.


\section{Conceptos teóricos}


Para llevar a cabo este proyecto, es fundamental comprender y aplicar diversos conceptos teóricos que sustentan tanto la adquisición y procesamiento de datos EEG como el desarrollo de modelos predictivos eficientes y precisos.

\subsection{Electroencefalograma}

El Electroencefalograma (EEG) es una técnica no invasiva utilizada para registrar la actividad eléctrica del cerebro. Las señales EEG son capturadas mediante electrodos colocados en la cabeza, que detectan los cambios en el potencial eléctrico generados por la actividad neuronal. Estas señales reflejan la actividad de gran cantidad de neuronas y son fundamentales para el estudio de diversas funciones cerebrales.

\imagen{memoria/eeg}{Ejemplo colocación electrodos para EEG.~\cite{egg:pixabay}}{.5}


Las principales señales EEG se componen de diferentes ondas que se clasifican según su frecuencia. Estas ondas reflejan distintos estados de actividad cerebral y son las que se utilizan para el análisis de datos de señales EEG.


\begin{itemize}

	\item
	\textbf{Ondas Delta:} 	
	\begin{itemize}
	
	\item
	\textbf{Frecuencia:} 0.5 - 4 Hz.
	\item 
	\textbf{Asociación:} Sueño profundo y estados de inconsciencia.
	\end{itemize}
	
	\item
	\textbf{Ondas Theta:}
	\begin{itemize}
	
	\item
	\textbf{Frecuencia:} 4 - 8 Hz
	\item 
	\textbf{Asociación:} Estados de somnolencia, meditación, y el sueño ligero.
	\end{itemize}
	
	\item
	\textbf{Ondas Alfa:}.
	\begin{itemize}
	
	\item
	\textbf{Frecuencia:} 8 - 13 Hz
	\item 
	\textbf{Asociación:} Estado de relajación y vigilia tranquila con los ojos cerrados.
	\end{itemize}

	\item
	\textbf{Ondas Beta:}
	\begin{itemize}
	
	\item
	\textbf{Frecuencia:} 13 - 30 Hz
	\item 
	\textbf{Asociación:} Estados de alerta, concentración activa, y actividad mental intensa.
	\end{itemize}
	
	\item
	\textbf{Ondas Gamma:}
	\begin{itemize}
	
	\item
	\textbf{Frecuencia:} 30 - 100 Hz
	\item 
	\textbf{Asociación:} Procesos cognitivos complejos, como la percepción consciente y el procesamiento de la información sensorial.
	\end{itemize}
\end{itemize}


\subsection{Preprocesado de datos}

El preprocesamiento de datos EEG es una fase crucial en el análisis de señales EEG, ya que asegura la calidad y la fiabilidad de los datos antes de su análisis y clasificación. A continuación se describen las técnicas específicas que se han aplicado en este proyecto para el preprocesamiento de los datos EEG:

\begin{itemize}

	\item
	\textbf{Modificación datos targets por perdida o unificación:}	
	
	
	Se ha realizado una unificación de los datos targets, para esto se han identificado que valores únicos tenia esta característica del conjunto de datos y una vez identificado unificar los que indican los mismo pero con otro valor. Como ejemplo: Left y LButton, son diferentes pero indican mismo valor.
	
	Los datos perdidos en target son aquellos que no indican ningún movimiento. Aunque no indiquen movimiento alguno se les ha de aplicar un valor para que  el análisis de los datos sea correcto. 


	\item
	\textbf{Eliminación de señales del conjunto de datos:}
	
	Se eliminan señales que no son necesarias para el análisis porque no aportan información relevante sobre la actividad cerebral

	
	\item
	\textbf{Eliminación de outliners:}
	
	Los outliers son valores atípicos que pueden distorsionar los resultados del análisis de datos. En el contexto de EEG, los outliers pueden surgir debido a artefactos o errores en la adquisición de datos.
	\begin{itemize}
	
	\item
	\textbf{Detección de Outliers:}
	
	Se aplican técnicas estadísticas para identificar valores que se desvían significativamente del resto de los datos. 
	
	
	El método utilizado ha sido z-core que mide la distancia de un valor desde la media del conjunto de datos en términos de desviación estándar.
	
	\begin{equation*}
	zcore = (dato - media.datos) / desviacion.estándar
	\end{equation*}

	
	Para la evaluación de que valores son outliners se ha utilizado la regla empírica o regla
	68-95-99.7.~\cite{outliners:empirica}

Usando esta regla, se consideran outliers los datos que tienen un z-score mayor a 3 o menor a -3, ya que caen fuera del rango en el que se encuentra el 99.7 por ciento de los datos.
	\end{itemize}

	\item
	\textbf{Escalado del conjunto de datos:}	
	Con el escalado de datos se prepararan los datos EEG antes de aplicar algoritmos o modelos de aprendizaje automático El objetivo es normalizar las amplitudes de las señales para que se encuentren dentro de un rango común, facilitando la comparación y el análisis.

Se ha aplicado un escalado en rangos de -1 a 1 para que la escala sea uniforme.

 
	\item
	\textbf{Segmentación de Datos:}	
	Se ha realizado segmentación de datos para dividir las señales EGG en ventanas temporales mas pequeñas asi se pueden identificar los experimentos a diferentes individuos realizados en el conjunto de datos.

\end{itemize}


\subsection{Partición de datos para el entrenamiento}

En machine learning y redes neuronales, es fundamental dividir el conjunto de datos en tres subconjuntos: entrenamiento (train), validación (val) y prueba (test). Cada uno de estos subconjuntos tiene un propósito específico y ayuda a garantizar que el modelo generalice bien a datos nuevos y no vistos con anterioridad en la ejecución de los modelos. 

Para lograr la división del conjunto de datos en subconjuntos he utilizado la función \textbf{train-test-split} de scikit-learn. Utilizándola en dos pasos.

1. División del conjunto de datos en Test y Train-intermedio. 90 por ciento para Train-intermedio.
2. División de Train-intermedio en Val y Train. 90 por ciento para Train.

Quedando la división aproximada en tanto por ciento, de la siguiente manera:
80 para Train, 10 para Val, 10 para Test

\begin{itemize}
	
\item
\textbf{Subconjunto de Entrenamiento (Train):}

	El subconjunto de entrenamiento se utiliza para ajustar los parámetros del modelo. Durante el entrenamiento, el modelo aprende patrones y relaciones en los datos de este subconjunto. En los experimentos sera del 80 por ciento de los datos del conjunto de datos inicial.

\item 
\textbf{Subconjunto de Validación (Val):}

	El subconjunto de validación se utiliza para ajustar los hiperparámetros del modelo y para evaluar su rendimiento durante el proceso de entrenamiento. Ayuda a prevenir el sobreajuste (overfitting). 
	
\item 
\textbf{Subconjunto de Prueba (Test):}


	El subconjunto de prueba se utiliza para evaluar el rendimiento final del modelo después de haber sido entrenado y ajustado utilizando los conjuntos de entrenamiento y validación. Se utilizan estos datos al final de la ejecucion del modelo.
	
	Este conjunto proporciona una medida final de la capacidad del modelo para generalizar a datos no vistos con anterioridad, lo cual es crucial para determinar si el modelo es apto para su implementación en un entorno real.
	
\end{itemize}



\subsection{Sobreajuste (overfitting)}

	El sobreajuste (overfitting) es un problema común en el aprendizaje automático y las redes neuronales, donde un modelo aprende demasiado bien los detalles y el ruido del conjunto de datos de entrenamiento, hasta el punto de que su rendimiento en datos nuevos y no vistos con anterioridad se deteriora. 
	
	El modelo se ajusta tan estrechamente a los datos de entrenamiento que pierde la capacidad de generalizar a otros conjuntos de datos.
	

\begin{itemize}	
\item
\textbf{Detección del sobreajuste:}

	Una indicación clara de sobreajuste es cuando la tasa de acierto del modelo en el conjunto de entrenamiento sigue mejorando mientras que la precisión en el conjunto de validación se estanca o empeora.

\textbf{Prevenir el sobreajuste:}

	\begin{itemize}
	\item
	\textbf{Dropout:}
	En redes neuronales, se desactivan algunas neuronas durante el entrenamiento para evitar que el modelo dependa demasiado de características específicas.
	
	Se utiliza después de las capas de entrada y capas ocultas de redes neuronales y los valores recomendados oscilan entre el 20 y 50 por ciento de neuronas desconectadas para que el aprendizaje no tienda al sobreajuste. Para este proyecto utilizare como valor, 30 por ciento.
	
	\item
	\textbf{Aumento de datos:}
	Generar nuevas muestras de datos a partir de las existentes mediante técnicas como la creación de ventanas deslizantes en series temporales o creación de nuevos datos sintéticos.
	
	\item
	\textbf{Validaciones cruzadas:}
	
	Dividir los datos en subconjuntos y entrenar el modelo varias veces, utilizando subconjuntos para entrenamiento y validación en cada iteración.
	
	\item
	\textbf{Uso de callbacks en redes neuronales:}
	
	Los callbacks permiten realizar acciones y monitorizar el entrenamiento en tiempo real, proporcionando una manera eficiente de controlar el modelo en su entrenamiento.
	
	Hay varios tipos de call back pero para el proyecto se utilizaran los siguientes:
	
		\begin{itemize}
		\item
		\textbf{EarlyStopping:}
	
	Detiene el entrenamiento cuando una métrica monitoreada deja de mejorar. En los modelos que se han implementado se utilizara la métrica \textbf{val-loss} (La pérdida (loss) es una medida de lo bien o mal que el modelo está haciendo sus predicciones en relación con los valores reales. Es una función matemática que calcula la discrepancia entre las predicciones del modelo y las etiquetas verdaderas.)
	
		\item
		\textbf{ReduceLROnPlateau:}	
	
	Monitoriza la misma métrica \textbf{val-loss} y reduce la tasa de aprendizaje en un factor de 0.1 si no mejora después de 5 épocas. 
	
		\item
		\textbf{ModelCheckpoint:}		
	
	Guarda el mejor rendimiento que ha tenido durante el entrenamiento, basado en la métrica \textbf{val-loss} y se guarda en un archivo .keras.
	

		\end{itemize}	
	
	\end{itemize}

\end{itemize}


\subsection{Machine Learning}

El uso de técnicas de machine learning en el análisis de datos EEG permite la detección y clasificación precisa de patrones en las señales cerebrales. En este proyecto, se han implementado diversos métodos de validación y algoritmos de clasificación para garantizar la robustez y fiabilidad de los modelos predictivos. 

A continuación, se describen las técnicas y algoritmos utilizadas:


\begin{itemize}

\item
\textbf{Técnicas de Validación:}

	\begin{itemize}
	
	\item
	\textbf{Holdout:}
	
	El método holdout implica dividir el conjunto de datos tal y como se ha descrito en la sección: "Partición de datos para el entrenamiento".
	
	
	Esta técnica es simple y rápida, los datos han de estar distribuidos uniformemente entre las particiones.
	
	Para que los datos estén distribuidos uniformemente dentro de los tres subconjuntos he utilizado el parámetro \textbf{stratify}, a este proceso se conoce como "estratificación" y asegura que el balanceo de los datos del conjunto original se mantenga en los subconjuntos de Train, Val y Test. Es importante porque permite que los modelos se entrenen y evalúen de manera más representativa.
	
	

	\item
	\textbf{K-Fold Cross-Validation:}
	
	
	La validación cruzada k-fold divide el conjunto de datos en k subconjuntos (folds) de tamaño aproximadamente igual.
	
	El modelo se entrena k veces, utilizando k-1 folds para entrenamiento y uno para prueba en cada iteración.
	
	Este proceso se repite k veces, asegurando que cada fold se utilice como conjunto de prueba una vez.
	
	\item
	\textbf{Leave-One-Out Cross-Validation:}
	
	Es una técnica de validación cruzada donde k es igual al número de muestras en el conjunto de datos.
	
	En cada iteración, una sola muestra se utiliza como conjunto de prueba, y el resto se utiliza para entrenamiento.
	
	Este método garantiza que cada muestra se evalúe como conjunto de prueba, proporcionando una evaluación precisa del modelo.
	
	Esta técnica puede ser computacionalmente costosa para conjuntos de datos grandes.
		
	\end{itemize}
	
	
\item
\textbf{Algoritmos de Clasificación:}	
	
	\begin{itemize}
	
	\item
	\textbf{K-Nearest Neighbors (KNN):}
	
	KNN es un algoritmo de clasificación basado en instancias que asigna una clase a una muestra en función de la mayoría de sus k vecinos más cercanos.
	
	La distancia entre muestras se calcula utilizando la distancia euclidiana.
	
	\imagen{memoria/KNN}{Uso KNN.~\cite{knn:intuitivetutorial}}{.7}
	
	

	\item
	\textbf{Árboles de Decisión:}	
	
	Los árboles de decisión son modelos de clasificación que dividen los datos en subconjuntos basados en valores de características, organizados en una estructura de árbol.
	
	Cada nodo del árbol representa una característica, cada rama una decisión, y cada hoja una clase.
	

	\imagen{memoria/TREE}{Uso arboles de decisión.~\cite{dibujos:visual}}{.5}	
	
	

	\item
	\textbf{Random Forest:}	
	
	Random Forest es un conjunto de árboles de decisión que utiliza el bagging (bootstrap aggregating) para mejorar la precisión y reducir el sobreajuste.
	
	Cada árbol del bosque se entrena con una muestra aleatoria del conjunto de datos, y las predicciones se combinan para obtener el resultado final.
	
	\imagen{memoria/RANDOM}{Uso random forest.~\cite{dibujos:visual}}{.5}	
	

	\end{itemize}
\end{itemize}
	

\subsection{Redes neuronales}

	
	
Las redes neuronales son modelos de aprendizaje profundo que simulan el funcionamiento del cerebro humano para reconocer patrones complejos en los datos.

En este proyecto, se han utilizado varios tipos de redes neuronales para el análisis de datos EEG:		
	
\begin{itemize}

	\item
	\textbf{Multilayer Perceptron (MLP):}

	Un MLP es una red neuronal artificial feedforward con una o más capas ocultas entre la capa de entrada y la capa de salida.
	
	Consta de una capa de entrada, una o más capas ocultas y una capa de salida. Cada neurona en una capa está conectada a todas las neuronas de la siguiente capa. Las capas ocultas permiten al MLP aprender representaciones no lineales de los datos. Tienen solo una dirección de conexión entre capas, hacia delante.
	
	El MLP se entrena utilizando un algoritmo que ajusta los pesos de las conexiones neuronales para minimizar el error de predicción.
	
	

	\imagen{memoria/MLP}{Red neuronal MLP.~\cite{dibujos:visual}}{.5}	
	

	\item
	\textbf{Recurrent Neural Network (RNN):}

	Una SRNN es un tipo de red neuronal en la que las conexiones entre las neuronas forman un ciclo, lo que permite que la información se mantenga a lo largo del tiempo.  Las neuronas de diferentes capas están conectadas hacia adelante o hacia atrás.

	Esta red tiene una capa de entrada, una o mas capas recurrentes ocultas y una capa de salida.
	
	
	\imagen{memoria/RNN}{Red neuronal RNN.~\cite{dibujos:visual}}{.5}	



	\item
	\textbf{Long Short-Term Memory (LSTM):}

	Las LSTM son un tipo de red neuronal recurrente diseñada para aprender dependencias a largo plazo en datos secuenciales. Utilizan celdas de memoria que pueden mantener información durante largos períodos.
	
	La arquitectura de esta red consiste en una capa de entrada, una o más capas LSTM y una capa de salida. 
	
	La red neuronal es muy parecida a la de las RNN pero las LSTM añaden:
	
	\textbf{Celdas} de memoria que sirven para escribir, leer o guardar datos, estas celdas están controladas por las puertas.
	
	\textbf{Puertas} que se dividen en purstas de entrada, de olvido y de salida que sirven para el control del flujo de datos, estas puertas regulan el estado de las celdas de memoria.


\end{itemize}


\subsection{Ventana Deslizante}

	
La ventana deslizante es una técnica que consiste en tomar un subconjunto de datos de un conjunto mayor, moviéndose a través del conjunto de datos por pasos fijos. Este subconjunto se llama "ventana". Cada vez que la ventana se desliza, se incluye un nueva ventana de datos que analizar. Para este proyecto se han creado ventanas deslizantes unificando temporalmente los datos por cada target, Key u objetivo.

Esto ayuda a manejar y analizar datos secuenciales, permitiendo al modelo enfocarse en patrones locales en los datos mientras mantiene la referencia a un target, Key u objetivo constante. Esto generara varios conjunto de ventanas, uno por cada objetivo.


\imagen{memoria/ventanas}{Ejemplo ventanas temporales sobre  targets.~\cite{dibujos:diagrams}}{0.95}		
	
	
\subsection{Generación de datos sintéticos}

	La generación de datos sintéticos en un conjunto de datos se refiere al proceso de crear nuevos datos que imitan las características estadísticas y estructurales del conjunto de datos original. 
	
	Este enfoque es útil en diversas situaciones, como cuando se quiere aumentar el tamaño del conjunto de datos para entrenar modelos de aprendizaje automático o equilibrar clases desproporcionadas.
	
	El método utilizado es sobremuestreo (Oversampling), principalmente replica y ajusta datos existentes para crear nuevas muestras. No elimina las muestras los datos originales sino que crea nuevos datos.

\subsection{Matriz de confusión}

Una matriz de confusión es una herramienta en la evaluación de modelos de clasificación en aprendizaje automático. Proporciona una visualización de la calidad del rendimiento de un modelo al comparar las predicciones del modelo con los valores reales de los datos de prueba.

\begin{itemize}

	\item
	\textbf{Verdaderos Positivos (VP):}
	
	Son los casos donde el modelo predijo correctamente la clase positiva.
	
	\item
	\textbf{Falsos Negativos (FN):}
	
	Son los casos donde el modelo predijo incorrectamente la clase negativa cuando en realidad era positiva.
	
	\item
	\textbf{Falsos Negativos (FP):}
	
	Son los casos donde el modelo predijo incorrectamente la clase positiva cuando en realidad era negativa.
	
	\item
	\textbf{Verdaderos Negativos (VN):}
	
	Representan los casos donde el modelo predijo correctamente la clase negativa.

\end{itemize}

\imagen{memoria/matriz}{Matriz de confusión.~\cite{dibujos:diagrams}}{0.6}	

\subsection{Tasa de acierto}

La tasa de acierto es una métrica que proporciona una visión general del rendimiento del modelo. Sin embargo, su interpretación debe realizarse con precaución, especialmente en casos donde las clases están desbalanceadas. Para este proyecto las clases están balanceadas.



	\begin{equation*}
	Tasa de acierto = (VP + VN) / VP + FN + FP + VN
	\end{equation*}
	
	Tasa de acierto es el número de predicciones correctas entre número total de predicciones.

\capitulo{4}{Técnicas y herramientas}


En esta sección de la memoria, además de detallar las técnicas metodológicas y las herramientas de desarrollo, se incluyen explicaciones sobre los diferentes experimentos realizados y las herramientas específicas utilizadas, como Jupyter Notebooks y Python. 

A continuación, se profundiza en estos aspectos:


\subsection{Experimentos realizados}



\subsubsection{Conjuntos de datos a analizar}

Se han creado cuatro conjuntos de datos, cada uno conteniendo señales EEG capturadas de diferentes sesiones o condiciones experimentales.

Estos conjuntos se han separado basándose en la característica TimeStamp dentro del conjunto de datos global.

Estos conjuntos están etiquetados por Segmentos, numerados del 1 al 4 y organizados de acuerdo con las especificaciones individuales para cada participante en el estudio.


Además de los conjuntos de datos individuales por sesión o condiciones experimentales, se han creado dos conjuntos de datos que combinan las señales EEG de todos las sesiones capturadas.

El primer conjunto se crea después de haber sido escalados los segmentos individualmente y se llama All Segmentos after.

El segundo conjunto se crea escalando todo el conjunto de datos sin haber separado los segmentos individualmente con anterioridad. All Segmentos before.


Un primer enfoque que permite analizar las diferencias entre los modelos y como clasifican los datos por sesiones o condiciones experimentales individualmente.

Y un segundo enfoque, mas atractivo, en un contexto unificado, proporcionando una visión generalizada de las clasificaciones de los modelos para varias sesiones o condiciones experimentales que aporta una visión mas real de estos análisis.

Total 6 conjuntos de datos para experimentar. 

4 para Segmentos individuales y 2 para conjuntos de datos "globales"


\subsubsection{Experimento comparativo}

En este experimento se compararon los modelos explicados en la sección Conceptos teóricos para modelos, machine learning y redes neuronales.

Se llevaron a cabo ejecuciones utilizando holdout, k-fold cross-validation , leave-one-out cross-validation, MLP, RNN y LSTM para evaluar la robustez y la generalización de los modelos desarrollados.

Se realizaron los experimentos con los 6 conjuntos de datos antes mencionados.







\capitulo{5}{Aspectos relevantes del desarrollo del proyecto}


El desarrollo de este proyecto siguió la metodología CRISP-DM (Cross Industry Standard Process for Data Mining) ~\cite{wiki:CRISP}, un enfoque estructurado y bien establecido para proyectos de minería de datos y análisis predictivo. 

A continuación, se detallan los aspectos más interesantes y relevantes de cada fase del ciclo de vida del proyecto según CRISP-DM.


\begin{itemize}
	
\item
\textbf{1.Comprensión del Negocio}

El objetivo principal fue identificar los objetivos y requisitos del proyecto desde una perspectiva empresarial.

Se llevaron a cabo reuniones con los tutores para entender los objetivos específicos del proyecto, como la detección y clasificación de acciones de movimiento (arriba, abajo, derecha, izquierda).

Se establecieron metas claras y medibles para el sistema de análisis de señales EEG, incluyendo los modelos a utilizar.


\item
\textbf{2.Comprensión de los Datos}

Unos de los objetivos principales fue familiarizarme con los datos disponibles y realizar un análisis preliminar.

Hubo una exploración inicial donde se exploro los conjuntos de datos EEG para comprender su estructura, características y distribución.

Se hizo uso de herramientas y visualizaciones para identificar patrones y posibles anomalías en los datos.

Se identificaron valores atípicos (outliners) que podrían afectar negativamente el rendimiento de los modelos.


\item
\textbf{3.Preparación de los Datos}

El objetivo principal de esta fase es preprocesar y preparar los datos para su análisis en los diferentes modelos.

En esta fase se crearon nuevos conjuntos de datos para luego realizar analisis sobre ellos. En total 6 como se ha descrito con anterioridad.

Se aplicación de técnicas de normalización y escalado, como StandardScaler y Z-Score, para estandarizar las señales EEG.

Creación y utilización de ventanas deslizantes, agrupandolas por cada uno de los tipos de datos de cada conjunto de datos generado.

Al final de la fase se propuso la generación de Datos Sintéticos para aumentar la cantidad y diversidad del conjunto de datos y abordar el análisis de los modelos desde otra perspectiva.

\item
\textbf{4.Fase de Modelado}

El objetivo en esta fase fue seleccionar y aplicar técnicas de modelado adecuadas. Los que se han utilizado para el proyecto son los siguientes:

\begin{itemize}
	
\item
\textbf{Algoritmos de Machine Learning:}

K-Nearest Neighbors (KNN)
Árboles de Decisión
Random Forest

\textbf{Evaluación y Comparación:}
Uso de métricas como Tasa de acierto y loss para evaluar el rendimiento de estos modelos.

\textbf{Redes Neuronales:}
Multilayer Perceptron (MLP)
Redes Neuronales Recurrentes (RNN)
Long Short-Term Memory (LSTM)

\textbf{Optimización y Regularización:} 

Uso de técnicas como Dropout para mejorar la generalización de los modelos.
\end{itemize}


\item
\textbf{5.Evaluación}


En esta fase se trata de poder evaluar el modelo para asegurar que cumple con los objetivos del negocio. Para ellos se utilizaran las siguientes técnicas:


\begin{itemize}
	
\item
\textbf{Métricas de Evaluación:}

Para todos los modelos se ha utilizado la evaluación mediante subconjuntos de datos de validación y prueba. (Val y Test)

\item
\textbf{Utilización de Callbacks y Técnicas de Monitorización:}

 Se ha hecho uso de EarlyStopping, ReduceLROnPlateau y ModelCheckpoint para evitar el sobreajuste y guardar el mejor modelo durante el entrenamiento.
\end{itemize}


\item
\textbf{6.Despliegue}

En esta fase el objetivo es la implementacion del proyecto, para ellos, describo a continuación las herramientas que se han utilizado:


\textbf{Jupyter Notebooks.} 

Se han utilizado no solo para el desarrollo iterativo y la experimentación rápida, sino también para la implementación modular del proyecto. 

La capacidad de crear, documentar y ejecutar celdas de código de manera interactiva permite a los usuarios finales poder ajustar y probar el proyecto en tiempo real, facilitando la depuración y el ajuste fino.

La naturaleza modular de los Jupyter Notebooks ha permitido separar distintas fases del proceso, desde la carga y preprocesamiento de datos hasta el entrenamiento y evaluación del modelo. 

Esta separación ha facilitado el mantenimiento y la actualización de cada componente sin afectar al resto del sistema.
\end{itemize}
\capitulo{6}{Trabajos relacionados}


El análisis de las señales EEG (Electroencefalografía) es un campo de investigación extenso y dinámico con numerosas aplicaciones en la neurociencia, la medicina y la interfaz BCI.

A pesar del considerable progreso en esta área, aún existen desafíos significativos que impiden alcanzar tasas de acierto consistentemente altas y lograr una relevancia práctica más amplia.

\subsection{Artículos científicos}

\begin{itemize}

\item
\textbf{Time Series Classification of Electroencephalography Data}

En este articulo ~\cite{Conference:paper} de 2023, se escribe sobre varios experimentos realizados con señales EEG, en concreto en la pagina 606, el experimento llamado \textbf{Hand Movement Direction}, se basa en que dos individuos movieran un joystick hacia arriba, abajo, izquierda o derecha según su elección.

Este articulo es el que mas se asemeja al proyecto actual, dando resultado de tase de acierto de como máximo 46,9 por ciento. Estos datos no son nada fructíferos al igual que el proyecto presentado.

\item
\textbf{EEG source imaging of hand movement-related areas: an evaluation of the reconstruction and classification accuracy with optimized channels}

En este articulo ~\cite{Article:springer} de Mayo de 2024 se detalla un experimento de clasificación de señales de dirección. A los individuos se les mostraba una flecha de dirección en una pantalla y ellos debían de pensar en esa dirección, se recogieron las señales EGG por interfaz BCI. El gran trabajo de preprocesado de las señales de EGG han dado resultados altos pero no óptimos.

Este experimento no es igual al proyecto que se presenta pero si representa el variado grupo de investigaciones que se están realizando en la actualidad sobre señales EEG.

La mayor tasa de acierto para este articulo ha sido de 84,4 por ciento.


\end{itemize}



\capitulo{7}{Conclusiones y Líneas de trabajo futuras}

\subsection{Conclusiones}


Este proyecto ha implicado un trabajo intenso y dinámico, caracterizado por su constante evolución y los desafíos que he ido enfrentando y superando a lo largo del camino. 

Ha requerido adaptarse a cambios continuos y resolver problemas de manera progresiva, lo cual ha enriquecido mi experiencia y habilidades en el proceso.

\begin{itemize}
	
	\item
	\textbf{Impacto de la Cantidad de Datos:}
	
	Los resultados obtenidos muestran que la falta de datos puede ser una limitación significativa en la obtención de altas tasas de acierto en el análisis de señales EEG. La disponibilidad de conjuntos de datos más grandes y variados podría mejorar la capacidad de generalización de los modelos y, por lo tanto, aumentar la precisión de las predicciones.
	
\imagen{memoria/TEST_comparador}{Ejemplo resultados Tasa de acierto primer experimento comparativo para subconjunto Test}{.9}	

	Las filas corresponden a los conjuntos de datos comentados en Aspectos relevantes del proyecto, Conjuntos de datos a analizar:
	
	Segmento del 1 al 4 son conjuntos de datos individuales por sesión o condiciones experimentales.
	
	All Segmentos after es conjunto de datos que se crea después de haber sido escalados los segmentos individualmente y unificados en un solo conjunto de datos.

	All Segmentos before es conjunto de datos que se crea escalando todo el conjunto de datos global sin haber escalado por separado los segmentos
	
	Las columnas de la tabla corresponden a los siguientes experimentos:
	
	KNN-TEST: Algoritmo de clasificación KNN y validación por Holdout.
	
	TREE-TEST: Algoritmo de clasificación por Arboles de decisión y validación por Holdout.

	RANDOM-TEST: Algoritmo de clasificación por random forest y validación por Holdout.

	MLP-TEST: Clasificación por red neuronal MLP.

	RNN-TEST: Clasificación por red neuronal RNN.

	LSTM-TEST: Clasificación por red neuronal LSTM.

	Se puede ver a simple vista que los resultados para la métrica Tasa de acierto no han sido buenos para cualquiera de los modelos comparados.

	Se han conseguido valores en Tasa de acierto de como máximo el 69,23 en uno de los segmentos individuales y del 36,76 en uno de los conjunto de datos completos.


	\item
	\textbf{Impacto del aumento de datos:}
	
	La inclusión de datos sintéticos ha demostrado ser beneficioso para mejorar las tasas de acierto en comparación con el uso exclusivo de datos reales. 
	
	Este enfoque ha permitido aumentar la diversidad y la cantidad de datos disponibles para el entrenamiento, lo cual es crucial cuando los datos reales son limitados.
	
	
\imagen{memoria/TEST_sinteticos}{Ejemplo resultados Tasa de acierto experimento con aumento de datos a Test}{.6}

	Las columnas de la tabla corresponden a los siguientes experimentos:

	RNN-RS-TEST: Clasificación por red neuronal RNN con aumento de datos.

	LSTM-RS-TEST: Clasificación por red neuronal LSTM con aumento de datos.

	Estos resultados en comparación con los anteriores son un 10 por ciento mayores pero aun no son buenos resultados para la métrica Tasa de acierto.
	

	\item
	\textbf{Impacto de las ventanas temporales en los análisis:}
	
	Los datos recogidos tras utilizar las ventanas temporales con datos reales y con el aumento de datos sintéticos han sido los siguientes:
	
\imagen{memoria/TEST_SW}{Ejemplo resultados Tasa de acierto experimento ventanas temporales a Test}{.6}

\imagen{memoria/TEST_SW_RS}{Ejemplo resultados Tasa de acierto experimento ventanas temporales con aumento de datos a Test}{.6}



	Las columnas de la tabla corresponden a los siguientes experimentos:

	RNN-SW-TEST: Clasificación por red neuronal RNN con ventanas deslizantes.

	LSTM-SW-TEST: Clasificación por red neuronal LSTM con ventanas deslizantes.

	RNN-RS-SW-TEST: Clasificación por red neuronal RNN con ventanas deslizantes y aumento de datos.

	LSTM-RS-SW-TEST: Clasificación por red neuronal LSTM con ventanas deslizantes y aumento de datos.

	
	El impacto combinado de las dos técnicas para el análisis hace que el aumento en el porcentaje de la tasa de acierto, para el conjunto de datos estandarizado por segmentos y luego unificado en un solo conjunto de datos, alcanza hasta el 83.64 por ciento.
	
	
	\item
	\textbf{Validación de datos Sintéticos contra datos Reales:}


	La comparación directa entre conjuntos con aumento de datos y subconjuntos Test de exclusivamente datos reales ha llegado a mostrar que los resultados  pueden llegar a superar significativamente, el rendimiento de Tasa de acierto que de los que solo eran datos reales. 
	
	Esto sugiere que los datos sintéticos pueden capturar mejor la variabilidad inherente en las señales EEG y mejorar la capacidad de generalización de los modelos.
	
	Unificando de nuevo las técnicas de aumento de datos y ventanas deslizantes se ha llegado hasta el 64,17 por ciento en Tasa de Acierto, algo esperanzador para futuras líneas de investigación puesto que mejora en un casi 30 por ciento de Tasa de acierto con los modelos comparativos con solo datos reales.

\imagen{memoria/TEST_real}{Ejemplo resultados Tasa de acierto experimento aumento de datos a datos reales Test}{0.6}

	Las columnas de la tabla corresponden a los siguientes experimentos:

RNN-RS-TEST: Clasificación por red neuronal RNN con aumento de datos.

LSTM-RS-TEST: Clasificación por red neuronal LSTM con aumento de datos.

RNN-RS-SW-TEST: Clasificación por red neuronal RNN con ventanas deslizantes y aumento de datos.

LSTM-RS-SW-TEST: Clasificación por red neuronal LSTM con ventanas deslizantes y aumento de datos.


\end{itemize}

\subsection{Líneas de trabajo futuras}


\begin{itemize}
	
	\item
	\textbf{Ampliación del Conjunto de Datos:}
	
	Una de las lineas de trabajo futuras sería la de seguir ampliando el conjunto de datos con más individuos voluntarios que colaboraran en realizar este tipo de experimentos para así poder formar un conjunto de datos mas extenso y representativo con el que poder seguir los análisis de clasificación.
	
	\item
	\textbf{Exploración de Datos Sintéticos y Ventanas Temporales:}
	
	La generación y uso de datos sintéticos ha mostrado ser efectiva en mejorar la Tasas de Acierto en los experimentos.
	
	Seria una buena linea de investigación futura. Se podría explorar y refinar técnicas de generación de datos sintéticos y segmentación temporal con ventanas deslizantes para capturar mejor la variabilidad y las relaciones temporales en las señales EEG.
	
	\item
	\textbf{Implementación de Análisis de Componentes Principales (PCA):}

	La implementación de PCA puede proporcionar una visión más profunda de las características más relevantes en las señales EEG, facilitando así la interpretación y el análisis de los datos. 
	Esto podría llevar a modelos más eficientes y comprensibles.	
	
	He creado un notebook llamado Preprocessing-Corr.ipynb que está junto al resto de los notebooks del proyecto, solo se puede ejecutar de manera Manual, he empezado a realizar análisis sobre la correlación de datos y creo que puede ser una buena linea que seguir.
	
\imagen{memoria/correlaciones}{Primeros pasos para lineas futuras de investigación en correlación}{.6}	
	
	
\end{itemize}	


\bibliographystyle{plain}
\bibliography{bibliografia}

\end{document}
