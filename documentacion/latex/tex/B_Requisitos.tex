\apendice{Especificación de Requisitos}

\section{Introducción}

La Especificación de Requisitos tiene como objetivo definir de manera clara y detallada las necesidades y expectativas del proyecto que consta de un notebook Jupyter definido en Python, el cual invoca a otros notebooks Jupyter secundarios para realizar diversas tareas. 

Este documento sirve para asegurar que todas las partes interesadas comprendan y acuerden los objetivos y funcionalidades del sistema a desarrollar.



\section{Objetivos generales}

Los objetivos principales de este TFG son los siguientes:

\begin{itemize}
\tightlist
\item
	\textbf{Facilitar el procesamiento de datos EEG.} Generados mediante un interfaz BCI.  	
\item
 	\textbf{Automatizar tareas repetitivas.} Con la creación de notebooks secundarios para poder ejecutar las tareas comunes y repetitivas relacionadas con los datos EEG. 
\item
	\textbf{Mejorar la tasa de acierto en los análisis de modelos} implementando modelos precisos y fiables. 
\item
	\textbf{Asegurar la accesibilidad y usabilidad en los notebooks.} Los notebooks son herramientas intuitivas y accesibles que aportan esa facilidad a la hora de interactuar.
\item
	\textbf{Integración y extensibilidad.} Al utilizar llamadas a otros notebooks se asegura que se permita la integración de nuevos notebooks y se asegura posibles nuevas mejoras en el código.
		
\end{itemize}



\section{Catálogo de requisitos}

Hay dos tipos de requisitos, los funcionales (qué debe hacer el sistema informático desarrollado) y los no funcionales (cómo debe funcionar el código): 

\subsection{Requisitos funcionales}

\begin{itemize}
\tightlist
\item
  \textbf{RF-001 Cargar archivos con datos EEG:}
 
  \begin{itemize}
  \tightlist
  \item
   \textbf{Descripción:} El sistema debe permitir cargar archivos de datos EEG para su posterior análisis.
  \item
   \textbf{Prioridad:} Alta
  \item
   \textbf{Criterios de aceptación:} El sistema debe permitir al usuario poder subir archivos con datos EEG manualmente. 
  \end{itemize}


\item
  \textbf{RF-002 Preprocesar datos:}
  
  \begin{itemize}
  \tightlist
  \item
    \textbf{Descripción:} El sistema debe preprocesar los datos EEG cargados de manera manual, estandarizar, escalar, eliminar outliners,... para su posterior análisis.
.
  \item
   \textbf{Prioridad:} Alta
  \item
   \textbf{Criterios de aceptación:} Se ha de poder comprobar que el sistema ha realizado el preprocesado de los datos EEG.
  \end{itemize}
 
  
\item
  \textbf{RF-003 Visualizar y analizar datos:}

  \begin{itemize}
  \tightlist
  \item
    \textbf{Descripción:} El sistema debe permitir la visualización gráfica de las señales EEG antes y después de ser preprocesadas.
  \item
    \textbf{Prioridad:} Alta
  \item
    \textbf{Criterios de aceptación:} Se ha de poder comparar los datos EEG antes y después del preprocesado. 
  \end{itemize}


\item
  \textbf{RF-004 Modelado y validación aprendizaje automático básico:}

  \begin{itemize}
  \tightlist
  \item
   \textbf{Descripción:} El sistema debe permitir entrenar modelos de aprendizaje básicos (KNN, Arboles de decisión, random forest) con diferentes técnicas de validación (Hold-Out,K-Fold, Cross-Validation, Leave-One-Out Cross-Validation) utilizando los datos EEG preprocesados.
  \item  
   \textbf{Prioridad: Alta}
  \item  
   \textbf{Criterios de aceptación:} .
	\textbf{}
  \end{itemize}


\item
  \textbf{RF-005 Modelado y validación aprendizaje automático neuronal:}
  
  \begin{itemize}
  \tightlist
  \item
   \textbf{Descripción:} El sistema debe permitir entrenar modelos de aprendizaje automático (MLP, SRNN, LSTM) utilizando los datos EEG preprocesados.
  \item  
   \textbf{Prioridad:} Alta
  \item  
   \textbf{Criterios de aceptación:} .
  \end{itemize}


\item
  \textbf{RF-006 Control de BCI:}

  \begin{itemize}
  \tightlist
  \item
   \textbf{Descripción:} El sistema debe ser capaz de identificar o clasificar con precisión las señales EEG que correspondan a una de las direcciones especificadas (arriba, abajo, izquierda, derecha).
  \item  
   \textbf{Prioridad: Alta}
  \item  
   \textbf{Criterios de aceptación:} .
  \end{itemize}


\item
  \textbf{RF-007 Almacenamiento y recuperación de datos:}

  \begin{itemize}
  \tightlist
  \item
   \textbf{Descripción:} El sistema debe permitir almacenar cada proceso tanto de manera temporal como permanente de los resultados experimentos y de la creación de archivos intermedios.
  \item  
   \textbf{Prioridad: Alta}
  \item  
   \textbf{Criterios de aceptación:} .
  \end{itemize}

\item
  \textbf{RF-008 Integración y Extensibilidad:}

  \begin{itemize}
  \tightlist
  \item
   \textbf{Descripción:} El sistema debe permitir la integración de nuevos modelos de aprendizaje automático sin necesidad de reestructurar significativamente el código existente.
  \item  
   \textbf{Prioridad: Alta}
  \item  
   \textbf{Criterios de aceptación:} .
  \end{itemize}


\end{itemize}


\subsection{Requisitos no funcionales}

\begin{itemize}
\tightlist
\item
  \textbf{RNF-001 Rendimiento:}
 
  \begin{itemize}
  \tightlist
  \item
   \textbf{Descripción:} El sistema donde se ejecuten los notebooks ha de poder manejar la ejecución sin causar demoras significativas.
  \item
   \textbf{Criterios de aceptación:} El tiempo de ejecución para la ejecución de cada notebook secundario no ha de ser superior a 15 minutos.
  \end{itemize}


\item
  \textbf{RNF-002 Usabilidad:}
  
  \begin{itemize}
  \tightlist
  \item
    \textbf{Descripción:} Los notebooks han de ser fáciles de entender y usar por los usuarios.
  \item
   \textbf{Criterios de aceptación:} La interfaz de los notebooks ha de ser intuitiva y clara.
  \end{itemize}
 
  
\item
  \textbf{RNF-003 Escalabilidad:}

  \begin{itemize}
  \tightlist
  \item
    \textbf{Descripción:} En la configuración del notebooks principal se debe permitir la adhesión de nuevos notebooks y sus llamadas desde el notebook principal.
  \item
    \textbf{Criterios de aceptación:} Poder seguir integrando nuevos notebooks para ser llamados desde el notebook principal sin afectar al funcionamiento del código. 
  \end{itemize}

\end{itemize}


\subsection{Restricciones}

\begin{itemize}
\tightlist
\item
  \textbf{R-001:} El sistema debe operar en un entorno Jupyter Notebook.
 
\item
  \textbf{R-002:} Todos los notebooks secundarios deben estar disponibles en el mismo entorno de ejecución que el notebook principal.

\item
  \textbf{R-003:} El procesamiento y análisis de datos debe realizarse utilizando bibliotecas compatibles con Python 3.x.

\end{itemize}




\section{Especificación de requisitos}

Voy a describir cada caso de uso identificado en el diagrama:


\begin{table}[p]
	\centering
	\begin{tabularx}{\linewidth}{ p{0.21\columnwidth} p{0.71\columnwidth} }
		\toprule
		\textbf{CU-001}    & \textbf{Carga archivo con datos EEG}\\
		\toprule		
		\textbf{Actor}              & Investigador    \\ 
		\textbf{Versión}              & 1.0    \\
		\textbf{Autor}                & José Luis Pérez Gómez \\
		\textbf{Requisitos asociados} & RF-001 \\
		\textbf{Descripción}          & El investigador carga el archivo de datos EEG para su posterior análisis. \\
		\textbf{Precondición}         & R-001, R-002, R-003\\
		\textbf{Acciones}             &
		\begin{enumerate}
			\def\labelenumi{\arabic{enumi}.}
			\tightlist
			\item El usuario siguiendo los pasos indicados desde el sistema, carga el archivo con datos EEG que quiere analizar.

		\end{enumerate}\\
		\textbf{Postcondición}        & El usuario podrá visualizar los datos cargados en el sistema\\
		\textbf{Excepciones}          &  \\
		\textbf{Importancia}          & Alta \\
		\bottomrule
	\end{tabularx}
	\caption{CU-001 Carga archivo con datos EEG.}
\end{table}



\begin{table}[p]
	\centering
	\begin{tabularx}{\linewidth}{ p{0.21\columnwidth} p{0.71\columnwidth} }
		\toprule
		\textbf{CU-002}    & \textbf{Preprocesamiento de Datos EEG}\\
		\toprule
		\textbf{Actor}              & Investigador    \\ 
		\textbf{Versión}              & 1.0    \\
		\textbf{Autor}                & José Luis Pérez Gómez \\
		\textbf{Requisitos asociados} & RF-002 \\
		\textbf{Descripción}          & El investigador ejecuta el preprocesado de los datos EEG. \\
		\textbf{Precondición}         & R-001, R-002, R-003 \\
		\textbf{Acciones}             & El sistema aplica técnicas de preprocesamiento para limpiar los datos, estandarizar, etc. \\
		\textbf{Postcondición}        & La salida de la ejecución no tenga ningún error\\
		\textbf{Excepciones}          &  \\
		\textbf{Importancia}          & Alta \\
		\bottomrule
	\end{tabularx}
	\caption{CU-002 Preprocesamiento de Datos EEG.}
\end{table}



\begin{table}[p]
	\centering
	\begin{tabularx}{\linewidth}{ p{0.21\columnwidth} p{0.71\columnwidth} }
		\toprule
		\textbf{Actor}              & Investigador    \\ 
		\textbf{CU-003}    & \textbf{Visualización de datos.}\\
		\toprule
		\textbf{Versión}              & 1.0    \\
		\textbf{Autor}                & José Luis Pérez Gómez \\
		\textbf{Requisitos asociados} & RF-003 \\
		\textbf{Descripción}          & El sistema genera visualizaciones gráficas de las señales EEG antes y después del preprocesado. \\
		\textbf{Precondición}         & R-001, R-002, R-003\\
		\textbf{Acciones}             &
		\begin{enumerate}
			\def\labelenumi{\arabic{enumi}.}
			\tightlist
			\item El sistema aplica técnicas de impresión por pantalla de los datos a analizar."
		\end{enumerate}\\
		\textbf{Postcondición}        & Visualizaciones gráficas de los datos EEG.\\
		\textbf{Excepciones}          &  \\
		\textbf{Importancia}          & Alta \\
		\bottomrule
	\end{tabularx}
	\caption{CU-003 Visualización de datos.}
\end{table}

\begin{table}[p]
	\centering
	\begin{tabularx}{\linewidth}{ p{0.21\columnwidth} p{0.71\columnwidth} }
		\toprule
		\textbf{CU-004}    & \textbf{Entrenamiento y validación de modelos de aprendizaje automático básico}\\
		\toprule
		\textbf{Actor}              & Investigador    \\ 
		\textbf{Versión}              & 1.0    \\
		\textbf{Autor}                & José Luis Pérez Gómez \\
		\textbf{Requisitos asociados} & RF-004 \\
		\textbf{Descripción}          & El investigador entrena y valida modelos de aprendizaje automático básico utilizando diferentes técnicas de validación con los datos preprocesados EEG. \\
		\textbf{Precondición}         & R-001, R-002, R-003\\
		\textbf{Acciones}             &
		\begin{enumerate}
			\def\labelenumi{\arabic{enumi}.}
			\tightlist
			\item El sistema entrena y valida los modelos aprendizaje automático básico con los datos preprocesados EEG.
		\end{enumerate}\\
		\textbf{Postcondición}        & Modelos entrenados y validados con sus respectivas métricas de tasa de acierto y matrices de confusión.\\
		\textbf{Excepciones}          &  \\
		\textbf{Importancia}          & Alta \\
		\bottomrule
	\end{tabularx}
	\caption{CU-004 Entrenamiento y validación de modelos de aprendizaje automático básico.}
\end{table}

\begin{table}[p]
	\centering
	\begin{tabularx}{\linewidth}{ p{0.21\columnwidth} p{0.71\columnwidth} }
		\toprule
		\textbf{CU-005}    & \textbf{Entrenamiento y validación de modelos de aprendizaje automático neuronal}\\
		\toprule
		\textbf{Actor}              & Investigador    \\ 
		\textbf{Versión}              & 1.0    \\
		\textbf{Autor}                & José Luis Pérez Gómez \\
		\textbf{Requisitos asociados} & RF-005 \\
		\textbf{Descripción}          & El investigador entrena y valida modelos de deep learning utilizando los datos EEG preprocesados.\\
		\textbf{Precondición}         & R-001, R-002, R-003, haber iniciado correctamente el notebook principal (Main) y haber ejecutado todas las celdas anteriores a esta celda\\
		\textbf{Acciones}             &
		\begin{enumerate}
			\def\labelenumi{\arabic{enumi}.}
			\tightlist
			\item El sistema entrena y valida los modelos aprendizaje automático neuronal con los datos preprocesados EEG.
		\end{enumerate}\\
		\textbf{Postcondición}        & Modelos entrenados y validados con sus respectivas métricas de tasa de acierto y matrices de confusión.\\
		\textbf{Excepciones}          &  \\
		\textbf{Importancia}          & Alta \\
		\bottomrule
	\end{tabularx}
	\caption{CU-005 Entrenamiento y validación de modelos de aprendizaje automático neuronal}
\end{table}


\begin{table}[p]
	\centering
	\begin{tabularx}{\linewidth}{ p{0.21\columnwidth} p{0.71\columnwidth} }
		\toprule
		\textbf{Actor}              & Investigador    \\
		\textbf{CU-006}    & \textbf{Control de BCI}\\
		\toprule
		\textbf{Versión}              & 1.0    \\
		\textbf{Autor}                & José Luis Pérez Gómez \\
		\textbf{Requisitos asociados} & RF-006 \\
		\textbf{Descripción}          & El sistema debe clasificar las señales EEG para determinar la dirección de movimiento (arriba, abajo, izquierda, derecha).\\
		\textbf{Precondición}         & R-001, R-002, R-003\\
		\textbf{Acciones}             &
		\begin{enumerate}
			\def\labelenumi{\arabic{enumi}.}
			\tightlist
			\item El sistema tras el entrenamiento y validación de los diferentes modelos debe ser capaz de identificar o clasificar con precisión las señales EEG que correspondan a una de las direcciones especificadas (arriba, abajo, izquierda, derecha)
			
		\end{enumerate}\\
		\textbf{Postcondición}        & \\
		\textbf{Excepciones}          &  \\
		\textbf{Importancia}          & Alta \\
		\bottomrule
	\end{tabularx}
	\caption{CU-006 Control de BCI}
\end{table}


\begin{table}[p]
	\centering
	\begin{tabularx}{\linewidth}{ p{0.21\columnwidth} p{0.71\columnwidth} }
		\toprule
		\textbf{Actor}              & Investigador    \\
		\textbf{CU-007}    & \textbf{Almacenamiento y Recuperación datos}\\
		\toprule
		\textbf{Versión}              & 1.0    \\
		\textbf{Autor}                & José Luis Pérez Gómez \\
		\textbf{Requisitos asociados} & RF-007 \\
		\textbf{Descripción}          & : El sistema almacena y recupera los archivos intermedios o resultados de los experimentos de análisis de datos EEG.\\
		\textbf{Precondición}         & R-001, R-002, R-003\\
		\textbf{Acciones}             &
		\begin{enumerate}
			\def\labelenumi{\arabic{enumi}.}
			\tightlist
			\item El investigador solicita la recuperación de resultados de los experimentos realizados y el sistema muestra los resultados de los experimentos.
			
		\end{enumerate}\\
		\textbf{Postcondición}        & Resultados de los experimentos almacenados y recuperados para análisis comparativo posterior.\\
		\textbf{Excepciones}          &  \\
		\textbf{Importancia}          & Alta \\
		\bottomrule
	\end{tabularx}
	\caption{CU-007 Almacenamiento y Recuperación de Resultados de Experimentos}
\end{table}


\begin{table}[p]
	\centering
	\begin{tabularx}{\linewidth}{ p{0.21\columnwidth} p{0.71\columnwidth} }
		\toprule
		\textbf{Actor}              & Desarrollador    \\
		\textbf{CU-008}    & \textbf{Integración y Extensibilidad del sistema}\\
		\toprule
		\textbf{Versión}              & 1.0    \\
		\textbf{Autor}                & José Luis Pérez Gómez \\
		\textbf{Requisitos asociados} & RF-008 \\
		\textbf{Descripción}          & : El desarrollador integra, modifica y extiende todos los modelos de aprendizaje automático en el sistema.\\
		\textbf{Precondición}         & R-001, R-002, R-003\\
		\textbf{Acciones}             &
		\begin{enumerate}
			\def\labelenumi{\arabic{enumi}.}
			\tightlist
			\item El desarrollador accede al código fuente del sistema, implementa el modelos,  actualiza la interfaz de usuario para permitir la selección de los modelos.
			
		\end{enumerate}\\
		\textbf{Postcondición}        & Modelos integrados y disponibles para experimentos de análisis de datos EEG.\\
		\textbf{Excepciones}          &  \\
		\textbf{Importancia}          & Alta \\
		\bottomrule
	\end{tabularx}
	\caption{CU-008 Integración y Extensibilidad del sistema}
\end{table}


