\apendice{Plan de Proyecto Software}

\section{Introducción}

La Universidad de Burgos, dentro del área de conocimiento de Ingeniería de Sistemas y Automática, dispone de un interfaz BCI (Brain Computer Interface) para la captación de señales cerebrales. 
Empleando ese interfaz se han realizado diferentes experimentos que han permitido recoger información de la actividad cerebral mientras los usuarios ejecutaban diferentes tareas cotidianas. 

Este Trabajo de Fin de Grado (TFG) tiene como objetivo principal el análisis de la información obtenida en esos experimentos. Se entrenarán diferentes algoritmos para clasificar la acción realizada por el usuario a partir de las señales generadas por el BCI. Con este propósito, se evaluarán diferentes algoritmos de procesado de señales y de machine/deep learning para la clasificación automática de señales.

El conjunto de datos para la realización del TFG son de tipo EEG (Electroencefalografía) referentes a experimentos basados en acciones sobre teclas de un teclado: arriba, abajo, izquierda, derecha.

El análisis de estos datos y su evaluación en diferentes algoritmos está basada en predecir qué teclas del teclado se han pulsado según las señales captadas con la interfaz BCI.

Para esto no ha habido una planificación como tal registrada en Github, pero sí una progresión definida en los commits de código generados durante la composición del TFG.

\section{Planificación temporal}


En la reunión inicial con los tutores definimos utilizar Python para realizar el código y una serie de aprendizajes básicos para poder acometer el TFG sin problemas.

En las siguientes reuniones se definieron algoritmos y experimentos a realizar con los datos EEG recogidos.

Desde el principio del proyecto debido a las circunstancias personales que como alumno tenía, no se ha podido realizar metodología Scrum, pero sí se produjeron estas lineas temporales que se muestran a continuación:

\imagen{anexos/Commits1}{Commits realizados durante la realización del TFG}

\imagen{anexos/Additions1}{Additions realizados durante la realización del TFG}

\imagen{anexos/Deletions1.png}{Deletions realizados durante la realización del TFG}


Se pueden dividir en 3 grandes etapas:

\begin{itemize}
\tightlist
\item
	\textbf{Etapa de estudio: (Febrero a Marzo)}   	
\item
 	\textbf{Etapa de desarrollo: (Abril a Mayo)} 
\item
	\textbf{Etapa de desarrollo y optimización: (Junio a Julio)} 
\end{itemize}





\subsection{Etapa de estudio: (Febrero a Marzo)}


\imagen{anexos/Commits-Etapa1}{Commits-Etapa 1}


\textbf{- Realización de cursos online} propuestos por los tutores Bruno Baruque y Jesús Enrique Sierra.
\href{https://www.kaggle.com/learn/pandas}{Tutorial de Inicio de Pandas}, \href{https://www.kaggle.com/learn/data-visualization}{Visualización de Datos}, \href{https://www.kaggle.com/learn/time-series}{Trabajo con series temporales}

\textbf{- Análisis del conjunto de datos}, en archivo csv, proporcionado por los tutores.

\textbf{- Creación esqueleto para la estructura el TFG en Github.}

\textbf{- Definición y creación de los primeros notebooks, principalmente análisis del conjunto de datos.}

\textbf{- Definición y creación de los primeros notebooks de machine learning}, al final de la etapa.

\textbf{- Comentarios en el código e impresión  de comentarios en los notebooks.}

Los principales problemas o obstáculos que me encontré fueron principalmente los siguientes:

\textbf{- Tiempo invertido en la realización de los cursos.}

\textbf{- Estructuración del código.} Me tomo mucho tiempo poder llegar a definir como quería mostrar el código, me decidí por un notebook principal que realizara llamadas al resto de notebooks con códigos mas específicos para cada modelo o experimento a realizar.

\textbf{- Impresiones por pantalla y definiciones básicas como normalizar o escalar el conjunto de datos.}



\subsection{Etapa de desarrollo: (Abril a Mayo)}


\imagen{anexos/Commits-Etapa2}{Commits-Etapa 2}

\textbf{- Cambio de análisis del conjunto de datos, prepocessing.}

\textbf{- Definición y creación notebooks de machine learning.}

\textbf{- Definición y creación notebooks de deep learning.}


Las dificultades que me encontré en esta etapa fueron:

\textbf{- Compilación modelos deep learning. }Al ejecutar los primeros modelos de deep learning los datos normalizados me proporcionaban errores de compilación con modelos SRNN o LSTM, cambiando a datos escalados y shapeando los modelos pudieron ejecutarse.

\textbf{- Utilización de ventanas temporales en los modelos deep learning.} En esta etapa no supe identificar este requerimiento por parte de los tutores y estuve implementando varias formas de poder utilizar ventanas temporales en el código.

\textbf{- Utilización de modelos deep learning y callbacks.} 

\textbf{- Gráficas y definiciones básicas como normalizar o escalar el conjunto de datos.}


\subsection{Etapa de desarrollo y optimización: (Junio a Julio)}


\imagen{anexos/Commits-Etapa3}{Commits-Etapa 3}

\textbf{- Añado nuevos gráficos} en análisis del conjunto de datos, prepocessing.

\textbf{- Definición y creación ventanas temporales acordadas con Bruno Baruque.}

\textbf{- Definición y creación nuevos notebooks de deep learning}

\textbf{- Definición y creación nuevos datos sintéticos a través de el aplicativo smote.}

\textbf{- Continuar con el comentado del código e imprimir comentarios en los notebooks.}


En la última etapa los problemas que se han acontecido son:

\textbf{- - Utilización de ventanas temporales en los modelos deep learning.} Después de varias algunas reuniones con Bruno Baruque se llego a la definición correcta para las ventanas temporal en el conjunto de datos.

\textbf{- Utilización de modelos deep learning y callbacks.}

\textbf{- Utilización datos sintéticos.} Definir correctamente esta generación de datos y poder utilizarlos en los modelos deep learning correctamente.




\section{Estudio de viabilidad}


\subsection{Viabilidad económica}

La viabilidad económica evalúa la posibilidad de generar ingresos y recuperar la inversión realizada a través del proyecto desarrollado. Se han de considerar las siguientes áreas:
  \begin{itemize}
   \tightlist
   \item
    \textbf{Potencial del proyecto para Otros BCIs:} 
    
    El código desarrollado para el análisis de señales EEG podría adaptarse y aplicarse a una variedad de sistemas BCI disponibles en el mercado. La flexibilidad del lenguaje de programación Python permite su integración con sistemas existentes, ampliando su potencial de mercado.
    
   \item
    \textbf{Mercado Objetivo:} 
    
    Los fabricantes de sistemas BCI constituyen el mercado principal para la comercialización del software desarrollado. Ejemplos de fabricantes incluyen NeuroSky, Emotiv, Brain Products y otros proveedores de dispositivos de interfaz cerebro-ordenador utilizados en investigación, medicina y aplicaciones comerciales. En España destacan Starlab Barcelona, Bitbrain, Neuroelectrics y Starlab.
    
   \item
    \textbf{Modelo de Negocio:} 
    
    Se propone un modelo de negocio basado en licencias de software. Los fabricantes de BCI podrían adquirir licencias comerciales del software para integrarlo en sus productos o servicios. Las licencias podrían estructurarse en diferentes niveles de funcionalidad y soporte, permitiendo adaptarse a las necesidades específicas de cada fabricante.
    
   \item
    \textbf{Estrategia de Comercialización:}
    
    La estrategia de comercialización incluiría demostraciones técnicas, pruebas piloto y colaboraciones estratégicas con fabricantes de BCI. Se realizarían  participaciones y colaboraciones en conferencias y eventos de la industria, así como la utilización de plataformas online para promover el software y captar el interés de potenciales clientes.

\end{itemize}



\subsection{Viabilidad legal}


Tipos de licencias para el sofware y bibliotecas instalado:

\begin{itemize}
 \tightlist
  \item
   \textbf{BSD (Berkeley Software Distribution):} 
    
    BSD es una licencia permisiva de código abierto que permite el uso, modificación y distribución del software con pocas restricciones. Requiere que se mantengan los avisos de copyright y las cláusulas de la licencia..
    
  \item
   \textbf{ASF (Apache Software License):}
    
    ASF 2.0 permite el uso, modificación y distribución del software, otorgando derechos de patente explícitos. Requiere mantener los avisos de copyright y proporcionar una NOTIFICACIÓN de cambios significativos.
    
  \item
   \textbf{Código abierto (Open source):}
    
    Las licencias de código abierto permiten el acceso, uso, modificación y distribución libre del software. Estas licencias fomentan la colaboración y la transparencia en el desarrollo de software, garantizando la libertad de uso y mejora.
\end{itemize}



El estudio de viabilidad ha demostrado que el desarrollo del software de análisis de datos EEG presenta una sólida oportunidad económica y cumple con los requisitos legales necesarios para su comercialización.

La aplicación potencial del software a otros sistemas BCI y la flexibilidad del modelo de negocio basado en licencias ofrecen un camino claro hacia la monetización y la expansión del proyecto.

  A continuación se presenta una tabla con algunas de las principales bibliotecas y herramientas de software utilizadas en el proyecto, junto con el tipo de licencia y las restricciones asociadas:
 

\begin{table}[p]
	\centering
	\begin{tabularx}{\linewidth}{ p{0.3\columnwidth} p{0.3\columnwidth} p{\columnwidth}}
		\toprule
		\textbf{Software}    & \textbf{Licencia} & \textbf{Restricciones} \\
		\toprule		
		\textbf{Anaconda}              & Gratuita    & Uso personal.  \\
		\toprule	
		\textbf{Jupyter Notebook}              & Libre    & Código abierto \\
		\toprule			
		\textbf{Python}              & Libre permisiva    & BSD \\
		\toprule		
		\textbf{Numpy}              & Libre permisiva    & BSD \\
		\toprule		
		\textbf{Scipy}              & Libre permisiva    & BSD \\
		\toprule		
		\textbf{Pandas}               & Libre permisiva    & BSD \\
		\toprule		
		\textbf{Matplotlib}              & Libre permisiva    & BSD \\
		\toprule		
		\textbf{Seaborn}              & Libre permisiva    & BSD \\
		\toprule		
		\textbf{Sklearn}               & Libre permisiva    & BSD \\
		\toprule		
		\textbf{Tensorflow}              & Libre permisiva    & ASF \\
		\toprule		
		\textbf{Ipywidgets}              & Libre    & Código abierto \\
		\bottomrule
	\end{tabularx}
	\caption{Licencias Software}
\end{table}