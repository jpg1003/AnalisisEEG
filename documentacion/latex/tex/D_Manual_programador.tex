\apendice{Documentación técnica de programación}

\section{Introducción}

Este TFG de análisis de un conjunto de datos EEG se centra en la aplicación de técnicas avanzadas de aprendizaje automático utilizando notebooks Python. El objetivo principal es procesar y analizar datos EEG para identificar patrones y correlaciones significativas que puedan revelar comportamientos específicos en la actividad cerebral a la hora de realizar un experimento el individuo.

El propósito fundamental es aplicar modelos de aprendizaje automático para mejorar la comprensión de las señales EEG, los datos aportados por la universidad de burgos constan de unos experimentos realizados con BCI y varios voluntarios es los que estos usuarios deben visualizar una serie de imagenes compuestas por direcciones, arriba, abajo, derecha e izquierda.

Este trabajo podría tener aplicaciones importantes en tecnologías de interfaz cerebro-computadora (BCI) aplicadas a por ejemplo,  control de sillas de ruedas, control de dispositivos electrónicos, realidad virtual, videojuegos, control de robots, etc...


\section{Estructura de directorios}

La estructura de los directorios del trabajo es la siguiente:

  \begin{itemize}
  \tightlist
  \item
   \textbf{/documentacion/:} Esta carpeta contiene toda la documentación relacionada con el TFG.
  \item
   \textbf{/documentacion/imagenes/:} Archivo de imágenes para todo el proyecto, documentación y código.
  \item
   \textbf{/documentacion/latex:} Documentación en formato latex y PDF.
  \item
   \textbf{/codigo:} Carpeta para archivar notebooks y datos para la ejecución del código.
  \item
   \textbf{/codigo/datos/:} Archivo datosEEGTotal.csv con el origen de datos.
  \item
   \textbf{/codigo/datos/csv/:} Archivos de datos csv auto generados tras las ejecuciones de algunos notebooks.
  \item
   \textbf{/codigo/notebooks:} Contiene los notebooks Jupyter y por lo tanto, el código.
  \end{itemize} 
  
  
  

\section{Manual del programador}

Este manual sera utilizado principalmente por las personas involucradas en futuros cambios, mejoras o nuevos desarrollos en el código.

Para ello, detallo las funciones de cada uno de los componentes :


\section{Compilación, instalación y ejecución del proyecto}

\section{Pruebas del sistema}
