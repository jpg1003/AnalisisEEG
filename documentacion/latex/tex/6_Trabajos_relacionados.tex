\capitulo{6}{Trabajos relacionados}


El análisis de las señales EEG (Electroencefalografía) es un campo de investigación extenso y dinámico con numerosas aplicaciones en la neurociencia, la medicina y la interfaz BCI.

A pesar del considerable progreso en esta área, aún existen desafíos significativos que impiden alcanzar tasas de acierto consistentemente altas y lograr una relevancia práctica más amplia.

\subsection{Artículos científicos}

\begin{itemize}

\item
\textbf{Time Series Classification of Electroencephalography Data}

En este articulo ~\cite{Conference:paper} de 2023, se escribe sobre varios experimentos realizados con señales EEG, en concreto en la pagina 606, el experimento llamado \textbf{Hand Movement Direction}, se basa en que dos individuos movieran un joystick hacia arriba, abajo, izquierda o derecha según su elección.

Este articulo es el que mas se asemeja al proyecto actual, dando resultado de tase de acierto de como máximo 46,9 por ciento. Estos datos no son nada fructíferos al igual que el proyecto presentado.

\item
\textbf{EEG source imaging of hand movement-related areas: an evaluation of the reconstruction and classification accuracy with optimized channels}

En este articulo ~\cite{Article:springer} de Mayo de 2024 se detalla un experimento de clasificación de señales de dirección. A los individuos se les mostraba una flecha de dirección en una pantalla y ellos debían de pensar en esa dirección, se recogieron las señales EGG por interfaz BCI. El gran trabajo de preprocesado de las señales de EGG han dado resultados altos pero no óptimos.

Este experimento no es igual al proyecto que se presenta pero si representa el variado grupo de investigaciones que se están realizando en la actualidad sobre señales EEG.

La mayor tasa de acierto para este articulo ha sido de 84,4 por ciento.


\end{itemize}


