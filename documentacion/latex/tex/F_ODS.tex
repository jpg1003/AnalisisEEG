\apendice{Anexo de sostenibilización curricular}

\section{Introducción}


Durante el desarrollo de este proyecto, he tenido la oportunidad de profundizar en los principios de sostenibilidad delineados por CRUE, los cuales proporcionan un marco esencial para integrar prácticas sostenibles en nuestras actividades académicas y prácticas. Esta experiencia no solo ha enriquecido mi comprensión teórica sobre el análisis de datos EEG, sino que también ha fortalecido mis habilidades prácticas para abordar los desafíos ambientales y sociales desde una perspectiva integradora y proactiva.


Desde el inicio del proyecto, a través de la investigación y el análisis crítico, he llegado a apreciar la interconexión entre los aspectos sociales y económicos de la sostenibilidad. Este enfoque me ha permitido entender que la sostenibilidad no se limita únicamente a la conservación del medio ambiente, sino que también implica asegurar la equidad social y promover la viabilidad económica a largo plazo.



\section{Aplicación práctica}

Una parte central de mi aprendizaje ha sido la aplicación práctica de los principios de sostenibilidad en este proyecto en concreto. A medida que avanzaba en la elaboración de estrategias y soluciones, tuve la oportunidad de evaluar el ciclo de vida de productos y procesos, identificar áreas de mejora y proponer medidas para minimizar nuestro impacto ecológico. Este enfoque práctico no solo me ha equipado con habilidades técnicas, sino que también me ha inculcado un sentido de responsabilidad hacia la integración de prácticas sostenibles en mi vida personal y profesional.

Además, el uso de licencias gratuitas o de uso libre permisivo para las herramientas y recursos utilizados en este estudio subraya el compromiso con la accesibilidad y la equidad. Esta elección no solo facilita la replicabilidad y la colaboración, sino que también promueve un acceso más amplio a los resultados y beneficios del proyecto.

\section{Pensamiento crítico}

Otro aspecto clave de mi desarrollo ha sido el fomento de un pensamiento crítico y creativo en relación con la sostenibilidad. A través del análisis de diversas perspectivas, he aprendido a cuestionar supuestos establecidos y a explorar soluciones innovadoras que puedan contribuir significativamente a la sostenibilidad global. Este proceso me ha enseñado a pensar de manera estratégica y a considerar no solo los desafíos inmediatos, sino también las implicaciones a largo plazo de nuestras acciones y decisiones.


\section{Impacto Personal}

Una faceta fascinante de este proyecto ha sido explorar cómo la tecnología de análisis de datos EEG puede mejorar significativamente la calidad de vida de las personas con discapacidades, como aquellas personas con discapacidad motora que utilizan sillas de ruedas. Al estudiar la posibilidad de utilizar conjunto de datos con señales EEG para ser utilizadas en la vida diaria o incluso en aplicaciones de entretenimiento como videojuegos accesibles, se puede llegar a vislumbrar un futuro donde la tecnología no solo facilita la inclusión, sino que también empodera a individuos con capacidades diversas para participar plenamente en la sociedad.

Este aspecto del proyecto no solo me ha inspirado personalmente, sino que también refuerza la importancia de considerar las implicaciones éticas y sociales de la tecnología en nuestro trabajo. La promoción de soluciones innovadoras y accesibles no solo beneficia a individuos con discapacidades, sino que también enriquece nuestra comprensión colectiva de cómo la tecnología puede ser utilizada para el bien común y el progreso social.


Este proyecto ha tenido un impacto profundo en mi compromiso personal. Me ha inspirado a adoptar un enfoque más consciente y reflexivo en mi vida diaria, buscando constantemente oportunidades para promover prácticas sostenibles en mi entorno. Además, me ha motivado a explorar carreras y oportunidades profesionales que me permitan contribuir de manera significativa a la construcción de un futuro más sostenible y equitativo para todos.

\section{Conclusión}

El proyecto no solo ha ampliado mis conocimientos y habilidades en sostenibilidad y tecnología de datos EEG, sino que también ha transformado mi perspectiva personal y profesional. Estoy convencido de que las competencias adquiridas aquí no solo serán valiosas en mi desarrollo académico y profesional, sino que también me capacitarán para enfrentar los desafíos globales con una mentalidad innovadora y comprometida con el cambio positivo. 

Espero seguir explorando y aplicando estas competencias a lo largo de mi vida, contribuyendo activamente a la creación de un mundo más sostenible y resiliente para las generaciones futuras.

