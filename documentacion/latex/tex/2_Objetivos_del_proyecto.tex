\capitulo{2}{Objetivos del proyecto}

-----------
Este apartado explica de forma precisa y concisa cuales son los objetivos que se persiguen con la realización del proyecto. Se puede distinguir entre los objetivos marcados por los requisitos del software a construir y los objetivos de carácter técnico que plantea a la hora de llevar a la práctica el proyecto.
---------------


El objetivo principal de este proyecto es diseñar, implementar y evaluar un sistema integral de análisis y clasificación de señales EEG (Electroencefalograma) centrado en la detección precisa y la clasificación efectiva de acciones de movimiento. 

Este sistema se construirá utilizando técnicas avanzadas de machine learning y redes neuronales, con el propósito de desarrollar modelos predictivos robustos capaces de interpretar y categorizar las señales cerebrales asociadas con movimientos específicos, como hacia arriba, abajo, derecha e izquierda.

El proyecto se enfocará en varias etapas clave:

\begin{itemize}
\tightlist
\item
\textbf{Adquisición y Preprocesamiento de Datos:}

 Se implementará un flujo de trabajo para la importación, limpieza y estandarización de los datos EEG, asegurando la calidad y consistencia necesarias para el análisis subsiguiente.

\textbf{Desarrollo de Modelos Predictivos:}

 Se explorarán y desarrollarán diferentes modelos de machine learning y redes neuronales, adaptados específicamente para el análisis de señales EEG. Esto incluirá el entrenamiento y la optimización de modelos para mejorar la precisión y la generalización.

\textbf{Validación y Evaluación:} 

Se llevará a cabo una evaluación exhaustiva de los modelos desarrollados, utilizando métricas como la tase de acierto. Además, se emplearán técnicas para garantizar la robustez de los resultados.

\end{itemize}

En resumen, el proyecto tiene como objetivo fundamental avanzar en la comprensión y aplicación de técnicas de machine learning y redes neuronales para el análisis de datos de señales EEG , específicamente en el contexto de para la detección precisa de acciones de movimiento.
