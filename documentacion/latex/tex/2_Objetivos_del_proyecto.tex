\capitulo{2}{Objetivos del proyecto}

El objetivo principal de este proyecto es diseñar, implementar y evaluar un sistema integral de análisis y clasificación de señales EEG (Electroencefalograma ) centrado en la detección precisa y la clasificación efectiva de acciones de movimiento. 

Este sistema se construirá utilizando técnicas avanzadas de machine learning y redes neuronales, con el propósito de desarrollar modelos predictivos robustos capaces de interpretar y categorizar las señales cerebrales asociadas con movimientos específicos, como hacia arriba, abajo, derecha e izquierda.


\section{Objetivos  técnicos}


El proyecto se enfocará en varias etapas clave:

\begin{itemize}
\item
\textbf{Adquisición y Preprocesamiento de Datos:}

 Se implementará un flujo de trabajo para la importación, limpieza y estandarización de los datos EEG, asegurando la calidad y consistencia necesarias para el análisis subsiguiente.

\item
\textbf{Desarrollo de Modelos Predictivos:}

 Se explorarán y desarrollarán diferentes modelos de machine learning y redes neuronales, adaptados específicamente para el análisis de señales EEG. Esto incluirá el entrenamiento y la optimización de modelos para mejorar la precisión y la generalización. Se emplearán técnicas como ampliar el conjunto de datos con la generación de nuevos datos sintéticos o la utilización de ventanas deslizantes para garantizar la robustez de los resultados.

\item
\textbf{Validación y Evaluación:} 

Se llevará a cabo una evaluación exhaustiva de los modelos desarrollados, utilizando métricas como la tasa de acierto.

\end{itemize}

En resumen, el proyecto tiene como objetivo técnico avanzar en la comprensión y aplicación de técnicas de machine learning y redes neuronales para el análisis de datos de señales EEG , específicamente en el contexto  para la detección precisa de acciones de movimiento.

\section{Objetivos  personales}


\begin{itemize}

\item
\textbf{Aprendizaje Avanzado en Machine Learning y Redes Neuronales:}

Adquirir experiencia práctica en la implementación y optimización de modelos complejos de machine learning y redes neuronales, aplicados específicamente al análisis de señales EEG.


\item
\textbf{Contribución al Avance Científico y Tecnológico:}

Contribuir al campo de la neurociencia computacional mediante el desarrollo de herramientas y técnicas que puedan mejorar la comprensión y el tratamiento de trastornos neurológicos basados en análisis de señales cerebrales.


\item
\textbf{Desarrollo de Habilidades en Investigación y Análisis:}

Mejorar habilidades en la investigación científica, la experimentación rigurosa y el análisis crítico de resultados.


\item
\textbf{Aplicación Práctica en Contexto Clínico o de Investigación:}

Facilitar la aplicación de conocimientos teóricos y técnicas avanzadas en un contexto práctico, contribuyendo a la mejora de del análisis de señales EEG.

\end{itemize}


