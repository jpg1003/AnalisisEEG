\capitulo{1}{Introducción}

El área de conocimiento de Ingeniería de Sistemas y Automática de la UBU dispone de un interfaz BCI (Brain Computer Interface) para la captación de señales cerebrales. Empleando ese interfaz se han realizado diferentes experimentos que han permitido recoger información de la actividad cerebral mientras un usuario ejecutaba diferentes tareas cotidianas. El TFG tiene como objetivo el análisis de la información obtenida en esos experimentos. Se entrarán diferentes algoritmos para clasificar la acción realizada por el usuario a partir de las señales generadas por el BCI. Con este propósito, se evaluarán diferentes algoritmos de procesado de señales y de machine learning para la clasificación automática de señales.


