\capitulo{1}{Introducción}

El análisis de señales de EEG (Electroencefalograma) es un campo en constante evolución con aplicaciones significativas en la Interacción Hombre-Máquina (HMI). 

Las señales de EEG, que capturan la actividad eléctrica del cerebro, ofrecen un medio no invasivo para entender y monitorizar los procesos cognitivos y motores humanos.


El análisis y clasificación de señales de EEG para detectar y categorizar acciones de movimiento representan un desafío de estudio significativo en la actualidad. 

Obtener la capacidad de desarrollar sistemas precisos y efectivos para esta tarea podría revolucionar la manera en que interactuamos con la tecnología, proporcionando interfaces más naturales e intuitivos. 

El objetivo de este proyecto es implementar un proceso de análisis que permita determinar de forma automática, con la mayor fiabilidad posible, la tecla de dirección del teclado que pulsa un usuario, dejando de lado análisis más complejos como por ejemplo el movimiento de un ratón, volante o mandos de juegos.


La Universidad de Burgos, dentro del área de Ingeniería de Sistemas y Automática, dispone de una interfaz BCI (Brain-Computer Interface) para la captación de señales cerebrales. 

Utilizando esta interfaz, se han realizado diferentes experimentos que han permitido recoger información de la actividad cerebral mientras los usuarios ejecutaban diversas tareas relacionadas con el movimiento.


Este Trabajo de Fin de Grado (TFG) tiene como objetivo el análisis de la información obtenida en esos experimentos. Se entrenarán distintos algoritmos para clasificar la acción realizada por el usuario a partir de las señales generadas por el BCI. 

Con este propósito, se evaluarán diferentes técnicas de procesamiento de datos, modelos de machine learning y redes neuronales para la clasificación de señales EEG, con el objetivo último de mejorar la detección y clasificación de movimientos específicos del usuario.








