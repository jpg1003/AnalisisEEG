\capitulo{1}{Introducción}

La Universidad de Burgos, dentro del área de conocimiento de Ingeniería de Sistemas y Automática, dispone de un interfaz BCI (Brain Computer Interface) para la captación de señales cerebrales. 
Empleando ese interfaz se han realizado diferentes experimentos que han permitido recoger información de la actividad cerebral mientras los usuarios ejecutaban diferentes tareas cotidianas. 

Este Trabajo de Fin de Grado (TFG) tiene como objetivo el análisis de la información obtenida en esos experimentos. Se entrenarán diferentes algoritmos para clasificar la acción realizada por el usuario a partir de las señales generadas por el BCI. Con este propósito, se evaluarán diferentes algoritmos de procesado de señales y de machine/deep learning para la clasificación automática de señales.

Los datos aportados son de tipo EEG (Electroencefalograma) para la realización del TFG son datos referentes a experimentos basados en acciones sobre teclas de un teclado: arriba, abajo, izquierda, derecha.

El análisis de estos datos y su evaluación en diferentes algoritmos esta basada en predecir qué teclas del teclado se han pulsado según las señales captadas con la interfaz BCI.


