\apendice{Especificación de diseño}

\section{Introducción}

La especificación de datos es un componente crítico en el desarrollo de sistemas de información, especialmente cuando se trabaja con conjuntos de datos complejos como los datos EEG. 

Mediante BCI, sistema que permite mediante adquisición de señales EEG, se han podido interpretar las señales adquiridas a través de las señales EEG y poder transformarlas en un conjunto de datos para su posterior análisis.  



\textbf{Descripción de los Datos:}

El archivo datosEEGTotal.csv contiene los datos facilitados para poder realizar los experimentos para la ejecución del TFG.

Su formato es de tipo CSV con un separador (;) entre los datos que lo componen.

Las características recogidas en el archivo de los datos EEG son las siguientes:

Timestamp, Attention, Meditation, Delta, Theta, LowAlpha, HighAlpha, LowBeta, HighBeta, LowGamma, HighGamma, Signal y Key.

\textbf{Timestamp:} Registro de tiempo para los experimentos, medido en mili-segundos.

\textbf{Attention:} Registra el grado de atención del participante que realiza el experimento.

\textbf{Meditation:} Grado de calma que tendría el individuo.

\textbf{Delta:} Son ondas de baja frecuencia (1 y 4 Hz), están presentes en etapas de sueño profundo, durante una meditación profunda y en pacientes con lesiones cerebrales o con TDAH severo.

\textbf{Theta:} Ondas entre 4 y 8 Hz, se encuentran en estados de calma profunda y sueño R.E.M., están ligadas al aprendizaje, memoria y intuición.

\textbf{Alpha:} Ondas entre 8 y 12 Hz, representan un estado de poca actividad cerebral y se asocian a un estado de calma mental. Divididas en dos señales LowAlpha y HighAplpha

\textbf{Beta:} Se diferencian en LowBeta y HighBeta, su frecuencia esta entre 12 y 35Hz, asociadas a una alta actividad mental. 

\textbf{Gamma:} En los datos se diferencia LowGamma y HighGamma, son ondas por encima de 30Hz y suelen aparecer cuando hay una alta concentración o atención

\textbf{Signal:} Podría ser la señal de que aporta la interfaz BCI.

\textbf{Key:} Valores target de lo que el individuo estaba pensando o visualizando durante el experimento.


La transformación de datos o el preprocessing que se ha realizado para poder afrontar los experimentos del TFG han sido los siguientes:


\textbf{Unificación de características Key:} El conjunto de datos tiene varios valores en Key que indican los mismo. LButton y Left.

\imagen{anexos/Key}{Cantidad de elementos para la característica Key}

\textbf{División del conjunto de datos:} El conjunto de datos tiene cuatro segmentos divididos por su Timestamp, superponiéndose entre ellos. Divido en estos cuatro segmentos para poder realizar experimentos y también dejo el conjunto de datos sin dividir para realizar experimentos conjuntos a los cuatro segmentos.

\imagen{anexos/Timestamp}{Análisis característica Timestamp del conjunto de datos}

\textbf{Eliminación de características:} Elimino las características Signal por no aportar nada significativo en el conjunto de datos y Timestamp porque no quiero que los datos aportados puedan tener una patrón temporal que haga que los experimentos no sean reales.


\textbf{Escalado de datos:} He utilizado la opción de escalar los datos ya que no son datos normales puesto que no tienen una distribución gaussiana en sus datos. 

\section{Diseño de datos}




\section{Diseño procedimental}



\section{Diseño arquitectónico}

